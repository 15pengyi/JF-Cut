
%-------------------------------------------------------------------------
\section{User Interaction}
\label{section user interaction}

We design a user interface for both images and videos.
It helps users mark foreground and background based on visibility, build an energy function according to the scenario, make local refinements and finally shows the graph cut results.

%-------------------------------------------------------------------------
\subsection{Labeling}

For videos, it is quite difficult for users to specify the foreground and background elements.
The conventional way is to mark the nodes frame by frame, which is time-consuming.
In our system, we provide users a WYSIWYG way to do labeling (both for images and videos).
The basic idea is to mark what users actually see based on visibility.
This technique is combined with a ray casting algorithm which requires users to specify a visibility threshold \cite{05BS, 11CM}.

%-------------------------------------------------------------------------
\subsection{Scenario-Driven Energy Function}

An \textit{Energy Function} is used to define the capacity of each edge in the graph, which measures the similarity of different subsets.
The generic form of the energy function is given by:

\begin{equation}
\label{equation energy function}
E = \lambda \sum R(p) + \sum B(p, q)
\end{equation}

where $R$ and $B$ denote the region properties and boundary properties respectively, $\lambda$ specifies a relative importance of the region properties versus the boundary properties.
We provide users with different ways to build the energy function for different scenarios.

\paragraph*{\textbf{Scalar Field Data}} we simplify the energy function introduced by Boykov-Kolmogorov \cite{04BK} which uses a probability distribution to measure the similarity and has a low accuracy.
We propose to focus on the continuity of a structure:

\begin{equation}
\label{equation bk_capacity}
C_{s}(p,q) = \left\{
    \begin{array}{ll}
    0 & p, q \in \mathcal{F} \bigcup \mathcal{B}\\
    exp(-\frac{(I_p - I_q)^2}{2\sigma^2}) & otherwise
    \end{array}
\right.
\end{equation}

\begin{equation}
\label{equation bk_excessflow}
E_{s}(p) = \left\{
    \begin{array}{ll}
    +E_{max} & p \in \mathcal{F}\\
    -E_{max} & p \in \mathcal{B}\\
    0 & p \in \mathcal{U}
    \end{array}
\right.
\end{equation}

where the nodes are divided into three groups: $\mathcal{F}$(foreground), $\mathcal{B}$(background) and $\mathcal{U}$(unknown).
$C_s$ denotes the capacity of the neighboring edge, $E_s$ denotes the excess flow of a node.
$I$ denotes the intensity and $\sigma$ can be estimated as "sampling error".

\paragraph*{\textbf{Vector Field Data}} We define the region and boundary terms similar to \textit{Lazy Snapping} which measures the similarity in the feature-space:

\begin{equation}
\label{equation lazy capacity}
C_{v}(p,q) = \left\{
    \begin{array}{ll}
    0 & p, q \in \mathcal{F} \bigcup \mathcal{B}\\
    \frac{1}{1 + \| \mathbf{v}_p - \mathbf{v}_q \|} & otherwise
    \end{array}
\right.
\end{equation}

\begin{equation}
\label{equation lazy excessflow}
E_{v}(p) = \left\{
    \begin{array}{ll}
    +E_{max} & p \in \mathcal{F}\\
    -E_{max} & p \in \mathcal{B}\\
    \lambda \frac{\min{\|\mathbf{c}_i^\mathcal{B}-\mathbf{v}_p \|} - \min{\|\mathbf{c}_i^\mathcal{F}-\mathbf{v}_p \|}}{\min{\|\mathbf{c}_i^\mathcal{B}-\mathbf{v}_p \|} + \min{\|\mathbf{c}_i^\mathcal{F}-\mathbf{v}_p \|}} & p \in \mathcal{U}
    \end{array}
\right.
\end{equation}

where $C_v$ denotes the neighbor-capacity, $E_v$ denotes the excess flow and is defined by the minimum distance to the center of a cluster which the current node belongs to.
These centers are computed by \textit{K-Means Clustering} which is also parallelized.
It helps us extract key features from the foreground and background, so that the noises can be effectively restrained and the accuracy is improved.
In this algorithm, the number of clusters $k$ is set to be 8, and the seed points are initialized using a histogram equalization technique.

Our system also supports features extraction, which can convert a scalar data into a multi-dimensional data \cite{10PC}.
These features include gradient, hessian matrix \cite{02KKH} and curvature \cite{03KWTM}.
