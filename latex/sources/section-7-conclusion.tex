
%-------------------------------------------------------------------------
\section{CONCLUSION}
\label{section conclusion}

In this paper, we introduce a parallel graph cut approach named JF-Cut to handle large data sets which has two main advantages:

\begin{enumerate}
\item improving the performance of graph cut for large images and videos using a GPU-based graph cut scheme based on jump flooding, heuristic push /relabel and convergence detection;
\item providing users an interactive graph cut based data segmentation interface that allows for intuitive data selection, interaction and segmentation for large data sets.
\end{enumerate}

We use a variety of data sets (both a benchmark and different large data sets) to evaluate the performance of our approach.
The results show that our method achieves a maximum 40-fold (139-fold) speedup over conventional GPU- (CPU-) based approaches and can be effectively used in different scenarios.
The source code of JF-Cut will be soon available at https://github.com/15pengyi/JF-Cut.

Our future work includes extending our approach to support unstructured data, optimizing the codes and making an independent library for end users to use.
